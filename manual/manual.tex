\documentclass[12pt]{article}
\usepackage[utf8x]{inputenc}
\usepackage[T1]{fontenc}
\usepackage[rounded]{syntax}
\usepackage{minted}
\usepackage{hyperref}
\usepackage[margin=1.4in]{geometry}
\usepackage{fancyvrb}
\usepackage{framed}
\usepackage[rounded]{syntax}
\usepackage{listings}

\renewcommand{\abstractname}{}

\newcommand{\bs}[2]{%
\begin{minipage}{5in}%
\begin{syntdiag*}[\left{#1}\right{#2}][5in]%
}
\newcommand{\es}{\end{syntdiag*}\end{minipage}}

\title{Manual for the Myrkvi language}
\author{Tumi Snær Gíslason \\ Instructor \& mentor: Snorri Agnarsson\\\newline Compilers 2014\\University of Iceland}

\begin{document}
\maketitle


\begin{abstract}
Myrkvi is a simple programming language based on Morpho.
\end{abstract}

\tableofcontents

\section{Introduction}
Myrkvi is a programming language made in the course \emph{Compilers} in the spring of 2014. It's a much simpler version of the programming language Morpho\footnote{\url{http://morpho.cs.hi.is/}}  made by Snorri Agnarsson.

It's grammar is simple and the functionality is limited so it can only handle the most basic tasks.


\section{Usage and installation}

Myrkvi is written in Java using the \emph{JFlex} and \emph{Byaccj} tools.
It communicates with Morpho by emitting Morpho assembly language which is then translated by Morpho.

The requirements for compiling and running a Myrkvi program are Java\footnote{\url{https://www.java.com/en/download/}},
Byaccj\footnote{\url{http://byaccj.sourceforge.net/\#download}},
JFlex.jar\footnote{\url{https://github.com/tumsgis/Myrkvi/blob/master/JFlex.jar}} and
morpho.jar\footnote{\url{https://github.com/tumsgis/Myrkvi/blob/master/morpho.jar}}

In Unix, having the requirements, you can set up the Myrkvi environment by using the \emph{makefile}

\begin{framed}
\begin{minted}[linenos=true]{text}
> make
> make test
\end{minted}
\end{framed}



\section{Syntax}
\subsection{Primitives}
\subsubsection{Comments}
\emph{\#} makes sure that the rest of the line is ignored.

\subsubsection{Keywords}
\emph{if}, \emph{elif}, \emph{else}, \emph{while}, \emph{def}, \emph{return}, \emph{var}, \emph{print}, \emph{println}, \emph{not}, \emph{and}, \emph{or}.

\subsection{Grammar}

%LITERAL
\begin{framed}
\synt{LITERAL}:\\
\indent \bs {>>-} {-><} 
\begin{stack} 
	<double>\\
	<int>\\
	<string>\\
	<true>\\
	<false>\\
	<null> 
\end{stack} 
\es
\end{framed}

%NAME
\begin{framed}
\synt{NAME}:\\
\indent \bs {>>-} {-><}
[a-zA-Z]([a-zA-Z]|{0-9})*
\es
\end{framed}

%whitespace
\begin{framed}
\synt{whitespace}: \\
\indent \bs {>>-} {-><}
\begin{stack}
	"' '"\\
	'$\backslash r$' \\
	'$\backslash t$' \\
	'$\backslash n$'
\end{stack}
\es
\end{framed}

%comment
\begin{framed}
\synt{comment}: \\
\indent \bs {>>-} {-><}
'\#' .*\$
\es
\end{framed}


%start
\begin{framed}
\synt{start}:\\
\indent \bs {>>-} {-><} 
<program>
\es
\end{framed}

%program
\begin{framed}
\synt{program}:\\
\indent \bs {>>-} {-><} 
\begin{rep}<function>\\ \end{rep} 
\es
\end{framed}

%function
\begin{framed}
\synt{function}: \\
\indent \bs {>>-} {-><} 
def <NAME> '(' <optnames>  ')' <body> 
\es
\end{framed}

%body
\begin{framed}
\synt{body}: \\
\indent \bs {>>-} {-><}
'{' <decls> <exprs> '}'
\es
\end{framed}

%optnames
\begin{framed}
\synt{optnames}:\\
\indent \bs {>>-} {-><} 
\begin{stack} 
	\\ 
	<dummynames> 
\end{stack} 
\es
\end{framed}

%dummynames
\begin{framed}
\synt{optnames}:\\
\indent \bs {>>-} {-><} 
\begin{stack} 
	<dummynames> ',' <NAME> \\ 
	<NAME>
\end{stack} 
\es
\end{framed}

%names
\begin{framed}
\synt{names}:\\
\indent \bs {>>-} {-><} 
\begin{stack} 
	<names> ',' <NAME> \\ 
	<NAME> \\ 
	<NAME> '=' <expr> 
\end{stack} 
\es
\end{framed}

%decls
\begin{framed}
\synt{decls}:\\
\indent \bs {>>-} {-><} 
\begin{stack} 
	\\ 
	<decl> ';' <decls> 
\end{stack} 
\es
\end{framed}

%decl
\begin{framed}
\synt{decl}:\\
\indent \bs {>>-} {-><} 
var <names> ';' 
\es
\end{framed}

%exprs
\begin{framed}
\synt{exprs}:\\
\indent \bs {>>-} {-><} 
\begin{stack} 
	<expr> ';' <exprs>\\
	 <expr> ';' 
\end{stack} 
\es
\end{framed}

%expr
\begin{framed}
\synt{expr}:\\
\indent \bs {>>-} {-><} 
\begin{stack} 
	<NAME>\\ 
	<NAME> '=' <expr> \\
	<NAME> '(' <optargs> ')' \\
	print <expr> \\
	println <expr> \\
	return <expr> \\
	<OPNAME> <expr> \\
	<expr> <OPNAME> <expr> \\
	<LITERAL> \\
	'(' expr ')' \\
	<ifexpr> \\
	not <expr> \\
	<expr> and <expr> \\
	<expr> or <expr> \\
	while <expr> <body>	 
\end{stack} 
\es
\end{framed}

%optargs
\begin{framed}
\synt{optargs}:\\
\indent \bs {>>-} {-><} 
\begin{stack} 
	\\ 
	<args> 
\end{stack} 
\es
\end{framed}

%args
\begin{framed}
\synt{args}:\\
\indent \bs {>>-} {-><} 
\begin{stack} 
	<args> ',' <expr>\\ 
	<expr> 
\end{stack} 
\es
\end{framed}

%ifexpr
\begin{framed}
\synt{ifexpr}:\\
\indent \bs {>>-} {-><} 
if <expr> <body> <elifexpr> <elseexpr>
\es
\end{framed}

%elifexpr
\begin{framed}
\synt{elifexpr}:\\
\indent \bs {>>-} {-><} 
\begin{stack} 
	\\ 
	elif <expr> <body> <elifexpr>
\end{stack} 
\es
\end{framed}

%elseexpr
\begin{framed}
\synt{elseexpr}:\\
\indent \bs {>>-} {-><} 
\begin{stack} 
	else <body>\\ 
	
\end{stack} 
\es
\end{framed}




\section{Definition}
\subsection{Values}
Floating point numbers,integers,strings,\emph{true},\emph{false} and \emph{null} are all represented the same way as in Morpho, so that they're convenient to use with the Morpho assembly language.
Every expression is \emph{true} except for \emph{false}, \emph{null} and 0.


\subsection{Variables}
The \emph{var} keyword is used to assign variables. The variables are weakly typed so for example this is a legal expression:
\begin{framed}
\begin{minted}[linenos=true]{text}
var x = 1;
x = true;
\end{minted}
\end{framed}


\subsection{Definition of expressions}

\subsubsection{Integers}
\begin{lstlisting}
{0-9}+
\end{lstlisting}

\subsubsection{Floating point numbers}
\begin{lstlisting}
{0-9}+\.{0-9}+([eE][+-]?{0-9}+)?
\end{lstlisting}

\subsubsection{Character}
Undefined.

\subsubsection{String}
\begin{lstlisting}
\"([^\"\\]|\\b|\\t|\\n|\\f|\\r|\\\"|\\\'|\\\\|
(\\[0-3][0-7][0-7])|\\[0-7][0-7]|\\[0-7])*\"
\end{lstlisting}

\subsubsection{List}
Undefined.

\subsubsection{return-expression}
return \synt{expr}\\
stops the activation of the current function and returns \synt{expr}.
If no return-expression is in a function then the last expression of the function is returned.

\subsubsection{Boolean expressions}
\emph{not} \\
A unary prefix operation,left associative, having the highest precedence of the booleans.\\\newline
\emph{and} \\
A binary operation having right associativity and the same precedence as the \emph{or} operation.\\\newline
\emph{or} \\
A binary operation having right associativity.\\\newline

Comparisons: \emph{<},\emph{>},\emph{==},\emph{<=},\emph{>=}\\\newline

All boolean expressions return either \emph{true} or \emph{false}. 

\subsubsection{Call expressions}
A function can be called simply by calling its name with the right amount of parameters.

For the Morpho compiler to translate the assembly language correctly there must be a function called \emph{main} present.

\subsubsection{Binary operations}
\emph{+,-,*,/ and \%} are all left associative and have the same precedence.

\subsubsection{Unary operations}
\emph{not}, \emph{+} and \emph{-}.

\subsubsection{if-expression}
The if-expression(\emph{if(b)...} where b is a boolean expression) is a control sequence, better described in the grammar rules above.

\subsubsection{while-expression}
The while-expression(\emph{while(b)...} where b is a boolean expression) is a control sequence, better described in the grammar rules above.



\section{Examples}

\subsection{Hello,world.}
\begin{framed}
\begin{minted}[linenos=true]{text}
def main()
{
	println "Hello,world.";
}
\end{minted}
\end{framed}

\subsection{Fibonacci}
\begin{framed}
\begin{minted}[linenos=true]{text}
def fibo(n)
{
	if(n < 2)
	{
		return 1;
	}
	else
	{
		return fibo(n-1) + fibo(n-2);
	};
}
\end{minted}
\end{framed}


\subsection{Fizz-Buzz}
\begin{framed}
\begin{minted}[linenos=true]{text}
def main()
{
	var end = 100;
	var iter = 1;

	while iter <= end
	{
		if (iter % 3 == 0) and (iter % 5 == 0)
		{
			print "FizzBuzz ";
		}
		elif iter % 3 == 0
		{
			print "Fizz ";
		}
		elif iter % 5 == 0
		{
			print "Buzz ";
		}
		else
		{
			print iter ++ " ";
		};
		iter = iter + 1;
	};
	println "";
}
\end{minted}
\end{framed}



\end{document}